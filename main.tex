%%%%%%%%%%%%%%%%%
% This is an sample CV template created using altacv.cls
% (v1.1.2, 1 February 2017) written by LianTze Lim (liantze@gmail.com). Now compiles with pdfLaTeX, XeLaTeX and LuaLaTeX.
% 
%% It may be distributed and/or modified under the
%% conditions of the LaTeX Project Public License, either version 1.3
%% of this license or (at your option) any later version.
%% The latest version of this license is in
%%    http://www.latex-project.org/lppl.txt
%% and version 1.3 or later is part of all distributions of LaTeX
%% version 2003/12/01 or later.
%%%%%%%%%%%%%%%%

%% If you need to pass whatever options to xcolor
\PassOptionsToPackage{dvipsnames}{xcolor}

%% If you are using \orcid or academicons
%% icons, make sure you have the academicons 
%% option here, and compile with XeLaTeX
%% or LuaLaTeX.
% \documentclass[10pt,a4paper,academicons]{altacv}
\documentclass[10pt,a4paper]{altacv}

%% AltaCV uses the fontawesome and academicon fonts
%% and packages. 
%% See texdoc.net/pkg/fontawecome and http://texdoc.net/pkg/academicons for full list of symbols.
%% 
%% Compile with LuaLaTeX for best results. If you
%% want to use XeLaTeX, you may need to install
%% Academicons.ttf in your operating system's font 
%% folder.


% Change the page layout if you need to
\geometry{left=1cm,right=9cm,marginparwidth=6.8cm,marginparsep=1.2cm,top=1.25cm,bottom=1.25cm}

% Change the font if you want to.

% If using pdflatex:
\usepackage[utf8]{inputenc}
\usepackage[T1]{fontenc}
\usepackage[default]{lato}

% If using xelatex or lualatex:
% \setmainfont{Lato}

% Change the colours if you want to
\definecolor{Mulberry}{HTML}{72243D}
\definecolor{SlateGrey}{HTML}{2E2E2E}
\definecolor{LightGrey}{HTML}{666666}
\colorlet{heading}{Sepia}
\colorlet{accent}{Mulberry}
\colorlet{emphasis}{SlateGrey}
\colorlet{body}{LightGrey}

% Change the bullets for itemize and rating marker
% for \cvskill if you want to
\renewcommand{\itemmarker}{{\small\textbullet}}
\renewcommand{\ratingmarker}{\faCircle}

%% sample.bib contains your publications
\addbibresource{sample.bib}

\begin{document}
\name{Jean-Paul JACQUOT}
\tagline{Senior Software Engineer, Mobile}
%\photo{2.8cm}{Globe_High}
\personalinfo{%
  % Not all of these are required!
  % You can add your own with \printinfo{symbol}{detail}
  \email{jacquot.jpaul@gmail.com}
  \phone{+33650003783}
  \mailaddress{26 rue de la poterne, Fontenay-Tresigny}
  \location{Ile-de-France, France}

  %% You MUST add the academicons option to \documentclass, then compile with LuaLaTeX or XeLaTeX, if you want to use \orcid or other academicons commands.
%   \orcid{orcid.org/0000-0000-0000-0000}
}

%% Make the header extend all the way to the right, if you want. Extend the right margin by 8cm (=6.8cm marginparwidth + 1.2cm marginparsep)
\begin{adjustwidth}{}{-8cm}
\makecvheader
\end{adjustwidth}

%% Provide the file name containing the sidebar contents as an optional parameter to \cvsection.
%% You can always just use \marginpar{...} if you do
%% not need to align the top of the contents to any
%% \cvsection title in the "main" bar.
\cvsection[page1sidebar]{Expériences et projets}

\cvevent{Enseignement superieur}{ESME Sudria}{depuis 2012}{Ivry-sur-Seine}
\begin{itemize}
\item Cours magistral développement mobile et langage objets
\item Encadrenent projets électroniques - micro-contrôleur 
\end{itemize}

\divider

\cvevent{Senior Software Engineer, Android}{Withings}{depuis  Janvier 2022}{Paris}
\begin{itemize}
\item Refactorisation/modularisation de l'application
\item Support et suivi des nouveaux devices
\item Support et suivi des nouveaux devices
\item Kotlin, Compose, Hilt, Gitlab, Figma
\end{itemize}

\divider

\cvevent{Senior Software Engineer, Android}{Schibsted/Leboncoin}{Janvier 2019 -- Mai 2021}{Paris}
\begin{itemize}
\item Mise en place du pair-à-pair Lbc, intégration des modes de livraison mondial relais et colissimo 
\item Création du module de paiement LBC, intégration du 3DS et 3DS2 et de la gestion des cartes bancaires
\item Kotlin/Java, MVVM, Dagger2, Gerrit, Jenkins, Amplitude, Adyen 
\end{itemize}

\divider

\cvevent{Lead Developper,  Android}{Kudoz}{ Juin 2017 -- Janvier 2019}{Paris}
\begin{itemize}
\item Prise en charge de l'application Kudoz, management d’une équipe de  de 4 devs. Refonte graphique Android/IOS
\item Migration vers Kotlin, MVP, Koin, UI/UX, tests (unitaires, ui, monkey)
\end{itemize}

\divider

\cvevent{Software Engineer et Fondateur , Flutter}{Jnvui}{depuis Mai 2021}{Fontenay-Trésigny}
\begin{itemize}
\item Mise en place d'une application interne pour une agence de communication Mali21
\item Mise en place d'une solution digital pour la société Borea Dental 
\item Accompagnement de projets ministériels au Mali pour la société Sini
\item Flutter (Android, IOS, Web), Intégration lib C++, Ftp/UDP, VLC
\end{itemize}


\divider

\cvevent{Software Engineer, Linux Embeded}{OpenWide/Smile}{Octobre 2012 -- Juin 2017}{Paris-Franckfort}
\begin{itemize}
\item Travail dans l'univers de l'open source, Contributions OpenSource sur plusieurs projets, à travers différentes missions 
\item AOSP, Linux (Arch, Debian, Gentoo), LFS, Temps réel, Yocto
\item Thales, Coyote, Deutsche bahn, Bubble, Engie, Sietcom, SoftBank
\end{itemize}


%\cvsection{Journée type}
%% Adapted from @Jake's answer from http://tex.stackexchange.com/a/82729/226
%% \wheelchart{outer radius}{inner radius}{
%% comma-separated list of value/text width/color/detail}
%\wheelchart{1.5cm}{0.5cm}{%
%  6/8em/accent!30/{Dodo}, 
%  3/8em/accent!40/"Morning routine",
%  8/8em/accent!60/Boulot,
%  2/10em/accent/Sport et loisirs,
%  5/6em/accent!20/Temps en famille
%  %
%}

\clearpage

%% If the NEXT page doesn't start with a \cvsection but you'd
%% still like to add a sidebar, then use this command on THIS
%% page to add it. The optional argument lets you pull up the 
%% sidebar a bit so that it looks aligned with the top of the
%% main column.
% \addnextpagesidebar[-1ex]{page3sidebar}


\end{document}
